\documentclass[aoas]{imsart}


\usepackage{booktabs}
\usepackage[utf8]{inputenc}
\usepackage[T1]{fontenc}

\usepackage{amsmath}
\usepackage{amssymb}
\usepackage{graphicx}
\usepackage{xspace}
\usepackage[noae]{Sweave}
\usepackage{natbib}

\DefineVerbatimEnvironment{Sinput}{Verbatim}{fontshape=sl,fontsize=\footnotesize,baselinestretch=0.88}
\DefineVerbatimEnvironment{Soutput}{Verbatim}{fontsize=\footnotesize,baselinestretch=0.88}
\DefineVerbatimEnvironment{Scode}{Verbatim}{fontshape=sl,fontsize=\footnotesize,baselinestretch=0.88}

\newcommand{\V}{\operatorname{Var}}
\newcommand{\E}{\operatorname{E}}
\newcommand{\R}{\textsf{R}\@\xspace}
\newcommand{\pkg}[1]{\texttt{#1}\@\xspace}
\newcommand{\code}[1]{\texttt{#1}\@\xspace}
\newcommand{\median}{\operatorname{median}}
\newcommand{\ie}{\emph{i.e.}\@\xspace}
\newcommand{\eg}{\emph{e.g.}\@\xspace}
\newcommand{\sd}{s.d.}
\newcommand{\bG}{\bold{G}}
\newcommand{\bY}{\bold{Y}}

\setkeys{Gin}{width=0.8\textwidth}


\chardef\bslash=`\\ % p. 424, TeXbook
\newcommand{\ntt}{\normalfont\ttfamily}
\newcommand{\cn}[1]{{\protect\ntt\bslash#1}}
%\newcommand{\pkg}[1]{{\protect\ntt#1}}
%\let\fn\pkg
%\let\env\pkg
%\let\opt\pkg
\hfuzz1pc % Don't bother to report overfull boxes if overage is < 1pc
\newcommand{\envert}[1]{\left\lvert#1\right\rvert}
\let\abs=\envert


\arxiv{math.PR/0000000}

\begin{document}

\begin{frontmatter}

\title{Statistical models for assessing
  agreement in method comparison studies with 
  heterogeneous random raters and replicate measurements}
\runtitle{Agreement for heterogeneous random raters}
% \thankstext{T1}{Footnote to the title with the `thankstext' command.}

\begin{aug}
  \author{\fnms{Claus Thorn}  \snm{Ekstrøm}\corref{asd}\thanksref{t2}
    \ead[label=e1]{ekstrom@sund.ku.dk}}
  \and
  \author{\fnms{Bendix} \snm{Carstensen}\ead[label=e2]{bxc@steno.dk}}
  \thankstext{t2}{To whom correspondance should be addressed.}

  \runauthor{C. T. Ekstrøm et al.}

  \affiliation{University of Copenhagen and Steno Diabetes Center}

  \address{Biostatistics, Department of Public Health, \\ Øster  Farimagsgade 5B,
    1014 Copenhagen, Denmark. \\ 
          \printead{e1}}

  \address{Steno Diabetes Center,\\ Niels Steensens Vej 2, 2820
    Gentofte. \\
          \printead{e2}}

\end{aug}


\begin{abstract}
    Agreement between methods for quantitative measurements are
  typically assessed by computing limits of agreement between pairs of
  methods and/or by illustration through Bland-Altman plots. We
  consider the situation where the observed measurement methods are
  considered a random sample from a population of possible methods,
  and discuss how the underlying linear mixed effects model can be
  extended to this situation. This is relevant when, for example, the
  methods represent raters/judges that are used to score specific
  individuals or items. In the case of random methods, we are not
  interested in estimates pertaining to the specific methods, but are
  instead interested in quantifying the variation between the methods
  actually involved making measurements, and accommodating this as an
  extra source of variation when generalizing to the clinical
  performance of a method. In the model we allow raters to have
  individual precision/skill and permit linked replicates (\ie, when
  the numbering, labeling or ordering of the replicates within items
  is important). Applications involving estimation of the limits of
  agreement for two datasets are shown: A dataset of spatial
  perception among a group of students as well as a dataset on
  consumer preference of French chocolate. The models are implemented
  in the \pkg{MethComp} package for \R.
\end{abstract}

%\begin{keyword}[class=MSC]
%\kwd[Primary ]{60K35}
%\kwd{60K35}
%\kwd[; secondary ]{60K35}
%\end{keyword}

\begin{keyword}
\kwd{Agreement}
\kwd{Limits of agreement}
\kwd{Methods comparison}
\kwd{Mixed models}
\kwd{Random raters}
\end{keyword}


\end{frontmatter}



%\keywords{Agreement; Limits-of agreement; Method comparison; Mixed models; Random raters\\
%\noindent\hspace*{-4.2pc} Supporting Information for this article is available from the author or on the WWW under\break \hspace*{-4pc} \underline{http://www.biostatistics.dk/agreement/ancona.pdf}
%}  %%% semicolon and fullpoint added here for keyword style

%% Information for the first author.
%\author{Claus Thorn Ekstr\o m \footnote{Corresponding author: {\textsf{e-mail: ekstrom@sund.ku.dk}}, Phone: +45-35327597, Fax: +45 3532 7907} \and Bendix Carstensen } 
%\author{Claus Thorn Ekstr\o m\affil{a}\corrauth\ and
%  Bendix Carstensen\affil{b}}

%\address{\affilnum{a}Biostatistics, Department of
%    Public Health, University of Copenhagen, Denmark\\
%  \affilnum{b}Steno Diabetes Center, Denmark}

%\corraddr{Claus T. Ekstrøm, Biostatistics, Department of
%    Public Health, Øster  Farimagsgade 5B,
%    1014 Copenhagen, Denmark. E-mail: \textsf{ekstrom@sund.ku.dk}}
  
%  $^1$ \footnote{Corresponding author: {\textsf{e-mail: ekstrom@sund.ku.dk}}, Phone: +45-35327597, Fax: +45 3532 7907} \and Bendix Carstensen$^2$ } 

%\address[]{Biostatistics, Department of Public Health,
%  University of Copenhagen, Øster  Farimagsgade 5B, 1014 Copenhagen, Denmark}
%%%%    Information for the second author
%\author[Carstensen]{Bendix Carstensen\inst{2}}
%\address[\inst{2}]{Steno Diabetes Center,  Niels Steensens Vej 2, 2820
%  Gentofte, Denmark}
%%%%    \dedicatory{This is a dedicatory.}
%\Receiveddate{zzz} \Reviseddate{zzz} \Accepteddate{zzz} 


%% maketitle must follow the abstract.
\maketitle                   % Produces the title.




% \noindent\hspace*{-4.2pc} Supporting Information for this article is available from the author or on the WWW under\break \hspace*{-4pc} \texttt{http://www.biostatistics.dk/agreement/ancona.pdf}


\section{Introduction}
Comparison of methods for quantitative data is concerned with how well
two methods agree on the measurement of an item. The interest is not
on testing a hypothesis that the mean of the the two methods are
identical, but on estimating the size and the standard deviation of
the difference between them.

Several authors have considered statistical models for assessing
agreement with continuous measurements
\citep{Barnhart2007,Rousson2002,BlanAltm:2007}.
\citet{Carstensen.2008b} considered a mixed effect model for computing
the prediction limits (often termed the \emph{limits of agreement})
for designs with replicate measurements on each item. This approach
provides a full statistical model that can compute the limits of
agreement in the presence of replicate measurements instead of relying
on summary measures. Moreover, it forces the investigator to focus on
the nature of replicate measurements: are they exchangeable within
each method by item stratum or only within items (\ie, the replicates
are linked).

Here we consider the situation where we are not interested in
determining the bias and agreement between two \emph{specific} methods
but are interested in the agreement between two random methods. Thus,
the methods included in the study are regarded as a random sample of
possible methods from a larger population of available methods; random
methods are relevant, for example, when a group of judges/raters are
asked to rate a set of items using a predetermined scale. This is a
common occurrence for example when medical doctors are asked to give
second or third opinions on measurements taken on patients. We are
interested in how well medical doctors in general agree on the scoring
of a particular condition, more so than determining how well specific
doctors compare to each other. However, the agreement among random
raters are not restricted to humans as measurement measurement
methods. A similar approach can be used in combination with, say,
machine learning where a large number of (random) regression trees
are generated and we wish to compare how well two randomly selected
regression trees predict the same value. Here the regression trees act
as raters and we wish to investigate how well two randomly chosen trees agree.

The focus on random raters (as opposed to fixed methods/raters) is not
new and is a often found in relation to the calculation of intraclass
correlation coefficients (ICC) for random factorial design
\citep{Shoukri2010,Kilem2012}. However, they are typically concerned
with estimating the ratio of the intraclass variation to the total
variance. Here, our focus is on limits of agreement and we present a
model that easily accommodates individual precision/skill for each
rater and handles unbalanced designs where raters may rate different
subsets of items.

Limits of agreement estimated for random raters mimics the
reproducibility coefficient  \citep{ISO1994}. The reproducibility
coefficient includes the variability among different laboratories,
when observations are carried out under reprocibility conditions at
different laboratories. In that sense, the reproducibility coefficient
also seeks to increase the variability by allowing the laboratories to
be different, but again we wish to extend that concept to situations with
individual skill levels and with unbalanced designs. 

Section \ref{sec:model} sets up the necessary mixed effect model that
accommodates random raters, and Section \ref{sec:replicate} extends
the model to the situation where replicate measurements from the
methods are available on each item. Section \ref{sec:repeatability}
discusses repeatability while the model is applied to two datasets
with random raters in Section \ref{sec:example}.

\section{Models for agreement among random raters}
\label{sec:model}

The traditional setup for comparison of measurements methods is one
where exactly one measurement is taken with each method on each item
(here, `item' can refer to, for example, an individual, a sample, or
an image). The limits of agreement for two methods are computed as the
prediction interval for the difference between future measurements
taken by the two methods on a new item or, equivalently as a
prediction interval for a measurement on an item by method B, given a
measurement by method A and \emph{vice versa}. The focus of such
studies is typically the comparison (and possibly prediction) between
a few specific methods of interest.

When multiple raters are compared, the raters play the role of
methods, but we are not necessarily interested in the difference between any two
specific raters. Instead we consider each rater as a ``random'' rater/judge/expert
from an (essentially infinite) population of raters. Thus, when we
compute the prediction limits for the difference between measurements
by two randomly chosen raters we should ensure that the variation
between the random raters is taken into account in the modeling and
that the variation is included in the computation of the prediction
interval. The situation with random raters arises for example in
medical situations where a medical doctor is asked to give a second
opinion on the measurement from a patient.

If we carry over the two-way analysis of variance model from the
traditional Bland-Altman setup we model the value for a
measurement taken on item $i$ by rater $m$ as:
\begin{equation} \label{mod1}
\begin{split}
Y_{mi} &= \mu_i + b_m + e_{mi}, \\
b_m &~ \sim N(0, \xi^2), \\
e_{mi} &~ \sim N(0, \sigma^2),
\end{split}
\end{equation}
where $\mu_i$ is the ``true'' value for item $i$, $b_m$ is a random
effect that models a random bias for measurements taken by rater $m$,
$\xi$ is the variation of biases between raters, and $\sigma$ is the
individual variation of rater $m$
\citep{Bland:Altman.1999,Carstensen.2008b}. Note that above we have
implicitly assumed that the individual variation for each rater is the
same --- we will relax this assumption below.

The ``true'' value for item $i$ is of course arbitrary; a constant may
be added to all $\mu_i$s provided it is subtracted from the $b_m$s;
but since we have constrained the mean of the $b_m$s to be 0, the
$\mu_i$s represent the average assessment of item $i$ by the set of
raters at hand. Note that since the raters (or more precisely, the
variation between them) is the focus of interest, the $\mu_i$s are
essentially nuisance parameters. Even if only one measurement per
rater and item is available it is possible to estimate all parameters
in model \eqref{mod1}, so in order to produce limits of agreement for
random raters in this simple setup we just need to estimate the mean rater-specific
variation across raters.

The limits of agreement for the difference between measurements taken
by two random raters ($m$ and $m'$) on the same item corresponds to
the prediction interval of the difference $Y_{mi} - Y_{m'i}$.  If
measurements by different raters are assumed to be independent we get
that the limits of agreement (prediction interval) for the
difference between the two raters on a new item is
\begin{equation}
0 \pm z \sqrt{\V(Y_{mi} - Y_{m'i})} = 0 \pm z \sqrt{\V(Y_{mi}) + \V(Y_{m'i})},
\end{equation}
where $z$ is the quantile corresponding to the desired level of the
prediction interval. Under model \eqref{mod1}, the 95\% limits of
agreement simplifies to
\begin{equation}
  0 \pm 1.96 \times \sqrt{2\times(\xi^2 + \sigma^2)} \approx 0 \pm 2.8
  \sqrt{\xi^2 + \sigma^2}. \label{eq:LoA1}
\end{equation}
In practice we will replace the parameters in \eqref{eq:LoA1} by their
corresponding estimates. The value 1.96 can of course be replaced by
a suitable quantile from the $t$ distribution; the normal convention
is to use 2 (which incidentally is the 97.5\% quantile of the $t$
distribution with 60 degrees of freedom). In the rest of this paper we
shall use 2 for convenience.


Model \eqref{mod1} can be extended to allow for raters to have
different skill/precision such that some raters can be very precise
(\ie, have low residual variation) while others can be less precise
(\ie, have high residual variation). The value for a measurement taken
on item $i$ by rater $m$ becomes
\begin{equation} \label{mod2}
\begin{split}
Y_{mi} &= \mu_i + b_m + e_{mi}, \\
b_m &~ \sim N(0, \xi^2), \\
e_{mi} &~ \sim N(0, \sigma_m^2),
\end{split}
\end{equation}
where $\sigma_m$ is the individual variation of rater $m$.  Allowing
for heterogeneity among raters extends the traditional modeling and
the variance for rater $m$ on item $i$ becomes
\begin{eqnarray*}
\V(Y_{mi}) & = & \xi^2 + \sigma_m^2
\end{eqnarray*}
where we let the individual residual variances, $\sigma_m^2$, follow
some distribution of the residual variances which has support on the
positive real numbers.
%to account for the heterogeneity of variances between raters.
Thus, the distribution of the residual variances is central for the
prediction of differences between raters; it represents the
distribution of ``skill'' among available raters.  If prior knowledge
about the precision distribution is available then that information
can be used to model the distribution of the residual variances.  In
the following we assume that we have no prior knowledge about the
precision and/or that we have too little information to be able to
verify any distributional assumptions about the residual variances
that we might have.

The law of total variance provides the variance of a measurement from
a \emph{randomly chosen rater} from the population when applied to a fixed
item:
\begin{equation}
\begin{split}
\V(Y_{mi}) &=  \V_\sigma\bigl(\E(Y_{mi}|\sigma_m^2)\bigr) +
              \E_\sigma\bigl(\V(Y_{mi}|\sigma_m^2)\bigr) \\
          &= \V_\sigma( \mu_i ) + \E_\sigma( \xi^2 + \sigma_m^2)\\
          &= \xi^2 + \E_\sigma(\sigma_m^2).
          \label{tot-var}
\end{split}
\end{equation}
$E_\sigma$ and $\V_\sigma$ represent means and variances based on the
underlying distribution of the residual variances.
We can estimate the variance for a randomly chosen rater using the empirical counterpart of \eqref{tot-var}:
%The empirical counterpart of \eqref{tot-var} is
\begin{equation}
  \begin{split}
   \widehat{\xi^2} + \widehat{\E(\sigma_m^2)}
   &\approx \widehat{\xi^2} + \frac 1M \sum_{m=1}^M \widehat{\sigma_m^2}
  % \\
 %  &\approx \hat{\xi}^2 + \left(\frac 1M \sum_{m=1}^M \hat{\sigma}_m\right)^2\\
   \label{approximation}
  \end{split}
\end{equation}
where $M$ is the number of raters available in the dataset.  Hence,
the estimate for the 95\% limits of agreement between two randomly
chosen raters becomes
\begin{equation}
  0 \pm 2 \times \sqrt{2\times\bigl( \widehat{\xi^2} +
      \frac 1M \sum_{m=1}^M \widehat{\sigma_m^2}
%      \median(\hat{\sigma}_m^2)
      \bigr)}
\end{equation}

Equation~\ref{tot-var} states that the variance of a measurement by a
randomly chosen rater is the sum of between rater variance and the
average within-rater variance. The estimate relies heavily on the
assumption that the sample of raters at hand is representative of the
population of future raters. Moreover, the within-rater variation
might be poorly determined if the number of raters in the study is
small.



Even without replicate measurements on each item it is possible to
estimate the individual residual variance for each rater as long as
each rater has scored at least two items, because model \eqref{mod1}
by the very nature of the randomness of raters must impose an
assumption of 0 average difference between raters. This means that the
estimate of the single rater's variation is strongly dependent on the
other raters' results, because it is essentially the variation around
the common means ($\mu_i$).
%
It also emphasizes the crucial nature of the assumptions behind
\emph{random} raters, a truly random sample of raters is required to
ensure that the empirical means across the raters at hand is a fair
approximation of the true population mean. To put it another way, the
generalizations are only valid for a population of raters of which
those in the sample at hand can be considered a random sample. And
predictions of precision for future ratings also require that the
sample of items for which the prediction is made is also a random
sample of the population of items.




The total variance for a random rater in the situation with
homogeneous individual variances, $\xi^2+\sigma^2$, resembles the
variance for a random rater in the situation with heterogeneous
individuals variation, $\xi^2 + \E_\sigma(\sigma_m^2)$. However, we
cannot generally use $\sigma^2$ as replacement for the mean individual
rater variance $\E_\sigma(\sigma_m^2)$ and then just use the simpler
homogeneous model, \eqref{mod1}, to estimate the variance
components. If the design is balanced, \ie, all raters have scored the
same number of items, then we can use \eqref{mod1} even in the
heterogeneous case. If the design is unbalanced then it is necessary
to estimate $\E_\sigma(\sigma_m^2)$ by $\frac 1M \sum_{m=1}^M
\widehat{\sigma_m^2}$ since otherwise we give increased weight to
  raters that have scored more items (see Figure~\ref{designs} for
  examples). Model \eqref{mod1} can only replace \eqref{mod2} if each rater
  has scored exactly the same number of items.
 

\begin{figure}[tb]
%  {\scriptsize
    \begin{center}
      \begin{tabular}{c@{\hspace*{.5cm}}cccccccccccccccccc} \toprule
 & \multicolumn{5}{c}{I} & & \multicolumn{5}{c}{II} & &
 \multicolumn{5}{c}{III} \\ \cmidrule{2-6} \cmidrule{8-12}
 \cmidrule{14-18} 
 & \multicolumn{5}{c}{Rater} & & \multicolumn{5}{c}{Rater} & &
 \multicolumn{5}{c}{Rater} \\
 Item & $A$ & $B$ & $C$ & $D$ & $E$ & & $A$ & $B$ & $C$ & $D$ & $E$ &  & $A$ & $B$ & $C$ & $D$ & $E$ \\ \cmidrule{2-6} \cmidrule{8-12}
 \cmidrule{14-18} 
    1 & $\bullet$ & $\bullet$ & $\bullet$ & $\bullet$ & $\bullet$ &  & $\bullet$ & $\bullet$ &  &  &  & & $\bullet$ & $\bullet$ & $\bullet$ & $\bullet$ &  \\  
    2 & $\bullet$ & $\bullet$ & $\bullet$ & $\bullet$ & $\bullet$ &  & $\bullet$ & $\bullet$ &  &  &  & & $\bullet$ & $\bullet$ & $\bullet$ & $\bullet$ &  \\  
    3 & $\bullet$ & $\bullet$ & $\bullet$ & $\bullet$ & $\bullet$ &  & $\bullet$ &  & $\bullet$ &  &  & & $\bullet$ & $\bullet$ & $\bullet$ & $\bullet$ &  \\  
    4 & $\bullet$ & $\bullet$ & $\bullet$ & $\bullet$ & $\bullet$ &  & $\bullet$ &  & $\bullet$ &  &  & & $\bullet$ & $\bullet$ & $\bullet$ & $\bullet$ &  \\  
    5 & $\bullet$ & $\bullet$ & $\bullet$ & $\bullet$ & $\bullet$ &  & $\bullet$ &  &  & $\bullet$ &  & & $\bullet$ & $\bullet$ & $\bullet$ &  & $\bullet$ \\  
    6 & $\bullet$ & $\bullet$ & $\bullet$ & $\bullet$ & $\bullet$ &  & $\bullet$ &  &  & $\bullet$ &  & & $\bullet$ & $\bullet$ & $\bullet$ &  & $\bullet$ \\  
    7 & $\bullet$ & $\bullet$ & $\bullet$ & $\bullet$ & $\bullet$ &  & $\bullet$ &  &  &  & $\bullet$ & & $\bullet$ & $\bullet$ & $\bullet$ &  & $\bullet$ \\  
    8 & $\bullet$ & $\bullet$ & $\bullet$ & $\bullet$ & $\bullet$ &  & $\bullet$ &  &  &  & $\bullet$ & & $\bullet$ & $\bullet$ & $\bullet$ &  & $\bullet$ \\  
\bottomrule 
\end{tabular}
      
   \end{center}
%  }
  \caption{Examples of designs with 8 items and 5 random raters. A dot
    indicates that an item was scored by the rater. I) Balanced design
    where each rater scores every item. II) ``Teacher design'' where a
    single expert, $A$, scores every item while trainees (raters
    $B$--$E$) each score a few items. III) ``Chief phycisians design''
    where a large number of medical students (raters $A$--$C$)
    examines every item/patient while a smaller group of chief phycisians
    (raters $D$--$E$) oversee the students as they do the rounds.}
  \label{designs}
\end{figure}


% {\footnotesize
% The model \eqref{mod1} is a two-way analysis of variance
% model with random row (\ie, \texttt{meth}-) effect. The simple version
% with identical residual variances between methods can be fitted with
% either \texttt{lmer} or \texttt{lme}, whereas there is no facility for
% stratum-specific variances implemented in \texttt{lmer}, \texttt{lme}
% must be used:
% <<>>=
% library( MethComp )
% data( Ancona )
% Ancona <- Meth( Ancona, 1,2,3,4 )
% library(lme4)
% print( lmer( y ~ item - 1 + (1|meth),
%              data=subset(Ancona,repl==1) ),
%        correlation=FALSE )
% print( lme( y ~ item - 1, random = ~ 1|meth,
%             data=subset(Ancona,repl==1) ) )
% print( lme( y ~item-1, random = ~1|meth,
%             weights = varIdent( form=~1|meth ),
%             data=subset(Ancona,repl==1) ) )
% @
% }

% When we wish to compare the agreement between two \emph{random} raters
% we need to take the variation in the individual rater precisions into
% account since we do not know which particular two raters we will
% use.

%Now, the outer variances and expectations might as well be computed under
%the distribution of $\sigma_m^2$ as under $\sigma_m$.
%So by this token, the empirical counterpart of this results can be either of:

\section{Models for agreement among random raters with multiple
  measurements}
\label{sec:replicate}

It is not uncommon to have situations where multiple measurements by
each rater on some items are available \citep{BlanAltm:2007}. Multiple
measurements for combinations of raters and items makes it possible to
estimate the repeatability of a random rater and provides better
estimates of the individual residual variations $\sigma_m^2$.

As mentioned above, even if raters are measuring each item only once,
we can still estimate the residual variation of each rater from model
\eqref{mod2}; it will simply be the variation around the common
item-means. This is where the situation with random raters differs
from comparing specific measurement methods. Because raters are
considered random, we must necessarily assume that the mean difference
between two raters is 0, and by that token that any deviation from 0
is random.  This means that the implicit assumption of randomly chosen
raters is heavily exploited in the case without replicate
measurements.

To assess the variability of the \emph{precision} of a random rater
it is mandatory to have multiple measurements of each rater for some
item(s); in that case observations are classified by replicate too, so
we need a more elaborate model.

If multiple measurements by each rater on an item exist then we can
use the following extension of the model \eqref{mod2} for the $r$th
measurement by rater $m$ on item $i$:
\begin{equation}
\begin{split}
Y_{mir} &= \mu_i + b_m + a_{ir} + c_{mi} + e_{mir}, \\
b_m &~ \sim N(0, \xi^2) \\
a_{ir} &~ \sim N(0, \omega^2) \\
c_{mi} &~ \sim N(0, \tau_m^2) \\
e_{mir} &~ \sim N(0, \sigma_m^2)\\ \label{mod-linked}
\end{split}
\end{equation}
Note that two new variance components have been added relative to the
simple model in Equation~\ref{mod2} and that $\mu_i$ is still fixed
since agreement is concerned with the variation in scores of one
specific new item. The first variance component, $\omega$, is the
variation between replication instances; as such it is in principle
irrelevant for the comparison of raters.  The second, $\tau_m$, is the
variation between items within each method --- a rater-specific
interaction with the items. It represents the variability of a rater
across items; that is how a specific rater's measurements varies
relative to the \emph{average} measurement by all raters on a given
item. Thus, this is a variance component whose size for the individual
rater is very strongly tied to the concept of randomly chosen raters,
in the sense that the estimates of $\tau_m$ will depend on the sample
of raters to a much larger degree than will the estimates of
$\sigma_m$ which are entirely estimates of the individual rater's
variation around his own measurement mean.

For situations with multiple measurements by each rater we can follow
the same arguments that led to \eqref{eq:LoA1} and get limits of
agreement between two randomly chosen raters as:
\begin{equation}
  \begin{split}
   0 \pm 2 & \times \sqrt{2\times\left( \hat{\xi}^2 +
      \frac 1M \sum_{m=1}^M (\hat{\tau}_m^2 + \hat{\sigma}_m^2)
                     \right)}
%  \\
%   0 \pm 2 & \times \sqrt{2\times\left( \hat{\xi}^2
%                       + \median( \hat{\tau}_m^2 ) +
%                        \median(\hat{\sigma}_m^2)\right)} \\
  \end{split}.
  \label{eq:LoA-linked}
\end{equation}
where we have taken the average item-by-rater variation between raters
into account. Note that the measurements by different methods are no longer
independent because of the term $a_{ir}$ but these terms
cancel out when computing $\V(Y_{mir}-Y_{m'ir})$, because indices $i$ and
$r$ are identical for the two terms.

However, repeated measurements come in two guises: exchangeable and
linked \citep{BlanAltm:2007,Carstensen.2008b}. 
%If the repeated
%observations on an item generally do not vary then they are
%\emph{exchangeable} and repeated observations of an item can
%essentially be freely exchanged and will give the same
%result. 
Exchangeable observations arise if the order of the
observations for a given combination of rater and item is
irrelevant. In model \eqref{mod2} this corresponds to having
$\omega^2=0$ since there is no common variation within items from replicate to replicate.

%Conversely, if the order of the repeated observation taken by a rater
%on a specific item matters then we say that the replicates are
%\emph{linked}. 

Replicates are \emph{linked} if the first replicate by all raters are
made at the same time (or under similar circumstances), and if the
second replicate by all raters are made at the same time too,
etc. This means that the \emph{numbering} (but not necessarily the
\emph{ordering}) of the replicates carries some information about
similar circumstance of the measurement. Linked replicates occur,
for example, when the measurements are taken over a longer period of
time, or if there is some inherent ``memory'' in the raters (\ie, they
can remember earlier scores on the same item). Linked repeated
observations correspond to having $\omega^2>0$.
%, or more generally
%having the item by replicate means equal to .

Formula \eqref{eq:LoA-linked} is used to estimate limits of agreement
for both exchangeable and linked repeated observations. However, for
exchangeable repeated observations we fit model \eqref{mod2} without
the variance component $\omega^2$ and thus the two setups might result
in different limits of agreement despite using the same formula.



%Replicates are called
%\emph{exchangeable} if this is not the case, that is if replicate
%numbering can be freely exchanged within a given combination of
%$(m,i)$.



\section{Repeatability}
\label{sec:repeatability}

The limits of agreement are not always the only issue of interest ---
the assessment of repeatability and reproducibility for either a specific or
randomly chosen rater
may be of interest in their own right. In particular, if a rater has
large variation between replicates on the same item then the
repeatability and agreement with other raters will be poor.
Repeatability can only be assessed when multiple measurements on the
same item by each
rater are available.

In typical assessment of \emph{specific} raters, the repeatability
coefficient for a method is defined as the upper limit of a prediction
interval for the absolute difference between two measurements by the
same rater on the same item under identical circumstances. 

When replicates are exchangeable, the difference between two replicate
measurements, $r$ and $r'$ taken by rater $m$ on item $i$ is
\begin{equation}
 Y_{mir} - Y_{mir'} = e_{mir}- e_{mir'}
\end{equation}
so the repeatability is based only on the residual standard deviation,
\ie, $2.8\sigma_m$. For linked replicates this difference becomes
\begin{equation}
 Y_{mir} - Y_{mir'} = a_{ir}- a_{ir'}  + e_{mir}- e_{mir'}, \label{eq:linked-repeat}
\end{equation}
and the variation between replicates taken on the same item should be
factored into the calculation of the repeatability coefficient which
then becomes $2.8\sqrt{\omega^2 +\sigma_m^2}$. Note that these two
results are conditional on rater $m$ since we considered specific raters.


In the case of \emph{random} raters, the repeatability coefficient has a
slightly different meaning because we cannot hinge it on estimates of
variances from any specific rater --- they will just be a random set
of variances. Instead the repeatability must refer to the
average/expected repeatability, or even, if we cling to the traditional
definition, the variability of the repeatability as we may expect to
see it in a sample of raters.

Thus for exchangeable replicates we get that the repeatability is 
$$ %\begin{equation]
    E_\sigma(2.8\sigma_m)    
$$ %\end{equation}
which is estimated by
\begin{equation}
    2.8\frac{1}{M}\sum_{m=1}^M\widehat{\sigma_m}, \label{repeatability-coefficient}
\end{equation}
when raters are allowed to have different variances and as $2.8
\sigma$, otherwise.  Likewise, for linked replicates we get
\begin{equation}
    2.8\frac{1}{M}\sum_{m=1}^M\sqrt{\widehat{\omega^2} + \widehat{\sigma_m^2}}.
\end{equation}



% A small repeatability coefficient corresponds to good
% repeatability (\ie, little variation among replicates) while a large
% repeatability coefficient corresponds to poor repeatability.
%Thus, if the standard deviation of a measurement is $\sigma$ then the
%repeatability coefficient becomes $2 \times \sqrt{2\sigma^2} \approx 2.8
%\sigma$.


The latter argument assumes that the variability between replication
occasions can be considered representative of future scenarios.  If
the replicates are taken under substantially different circumstances,
then the variance component $\omega$ may be considered irrelevant for
the repeatability and the repeatability coefficient should be based on
the measurement errors alone, \ie, use
$2.8\frac{1}{M}\sum_{m=1}^M\widehat{\sigma_m}$.  However, if indeed the
replicates are taken under substantially different circumstances it
may be argued that we are not really measuring the same item
repeatedly. Instead the effects of differing replication circumstances
could be modeled by a systematic effect. Hence there is no
subject-matter-free way of defining repeatability from the variance
components in the models.

%Thus, repeatability varies between raters too, and in the case of
%random raters essentially only the median of these makes sense as a
%summary estimate --- half of the times one will have a rater with
%better repeatability, half of the times one with worse.

%\textsf{XXX : Jeg vil egentlig hellere skrive nedenstående. Måske er
% det bare fordi jeg ikke helt forstår argumentet ovenfor. Hvis vi
% vælger helt at fjerne medianen skal vi også huske at rette koden til
% i repeatabilityafsnittet i Ancona.Rnw:
%Thus, repeatability varies between raters too, so the mean of the
%repeatability coefficients makes sense as a summary estimate (even
%though the mean coefficient may not coincide with that of any
%particular rater found in the dataset).
%}
% \clearpage

\section{Example applicaitons}
\label{sec:example}


\subsection{Spatial perception of point swarms}

Some people have keen spatial perception and are able to almost
instantaneous\-ly give a reasonable guess of, for example, the number of
individuals in a crowd. We collected data on spatial perception from
participants attending a course on comparison of measurement
methods. Thus, we will use attendees of the course as a random
selection of raters and will try to determine how well two random
raters agree on assessing the number of points in a point swarm.

% \input{Ancona}


% \usepackage[utf8]{inputenc}



% \subsection{The dataset}
Ten pictures of scattered points were generated, and the purpose of
the exercise was to estimate the number of points in each picture
after having viewed the picture for 5 seconds.  The number of points
in each picture were between 24 and 120. Each picture were shown to
the 17 raters three times (rotated and in random order to prevent
recollection of the pictures), without telling them about the
replication structure.  The dataset is available as the dataset
\code{Ancona} (collected at a pre-confe\-rence course at the VIth
conference of Societ\`{a} Italiana di Statistica Medica ed
Epidemiologia Clinica (SISMEC) held in September 2011 in Ancona,
Italy) found in the \pkg{MethComp} package for \R.




We can get an overview of the data by plotting scores of the raters
(and the replicates within raters) for each item (picture) as shown in
Figure~\ref{fig:ancona-data}, separately for original scale and
log-transformed data.  The plots in Figure~\ref{fig:ancona-data}
provided pretty clear indications that a log-transform might provide a
better fit to the data, since the variances are increasing for the
analysis on the original scale, but largely stable for the
log-transformed.

\begin{figure}%[tbp]
  \centering
\includegraphics[width=0.9\textwidth]{Ancona-xyplot-lin}
\includegraphics[width=0.9\textwidth]{Ancona-xyplot-log}
\caption{Illustration of the Ancona data. Each panel shows a specific
  item and the columns of dots within each panel represent the scores
  given by each rater (with the true number of points listed above
  each panel). The symbols/colouring corresponds to replicates (with
  circles, triangles, and diamonds corresponding to first to third
  replicate, respectively). Raters are listed in the same order within
  all panels and are sorted according to their average log-score
  across all items and replicates. The upper graph shows counts on the
  original scale, the lower after log-transform. It appears that
  variances are reasonably homogeneous in the lower panels, but not in
  the upper. The solid horizontal lines correspond to the mean score
  within replicates, the dashed line to the true number of points.}
\label{fig:ancona-data}
\end{figure}

% We can get estimates of the variance components using the
% \code{BA.est} function on the \code{Meth} object \code{Ancona},
% where the \code{random.raters} argument is set to \code{TRUE} in
% order to consider methods/raters as random effects instead of fixed
% effects:
% XXXX skal det her med igen?
% What is printed by the last command is merely the estimated variance
% components using the model where the conversion between methods have
% intercept is 0 and slope 1. 

% is stored in the \code{LoA} element returned by
% \code{BA.est}, so the limits are (which we save for later use)

Using formula \eqref{eq:LoA-linked} we get that the limits of agreement between two
random raters is $\pm$68.02, thus
there is a 95\% probability that two randomly selected raters differ
less than 68 points in their
assessment --- not impressive since the range of values are from 24 to
120.



% If we wish to use the median of the variance components instead of the
% mean to get a more robust estimate of the limits of agreement then we
% have to extract the median of the relevant variance components and
% use them together with formula \eqref{eq:LoA-linked}. We also make the
% computations using the mean, illustrating the calculation of the LoA
% embedded in \texttt{BA.est}:
% <<>>=
% ( RRmd <- apply( RR$VarComp, 2, median ) )
% sqmean <- function( x ) sqrt(mean(x^2))
% ( RRmn <- apply( RR$VarComp, 2, sqmean ) )
% @ %def
% According to formula \eqref{eq:LoA-linked} we get the LoA as plus/minus:
% <<>>=
% ( LoA.md <- 2 * sqrt( 2 * ( RRmd["M"  ]^2 +
%                             RRmd["MxI"]^2 +
%                             RRmd["res"]^2 ) ) )
% ( LoA.mn<- 2 * sqrt( 2 * ( RRmn["M"  ]^2 +
%                             RRmn["MxI"]^2 +
%                             RRmn["res"]^2 ) ) )
% @ %def
% We get slightly smaller limits of agreement when we use the
% median of the estimated variance components instead of the mean.

% \subsubsection{Log-transformed data}

%This can be visualized by using the \code{DA.reg} function to make
%regressions of the averages on the means (assuming intercept and slope
%to be 0, thus essentially looking at the differences within all pairs
%of raters).
%<<trsf,fig=TRUE,height=5,width=10>>=
%par(mfrow = c(1, 2), mar = c(3, 3, 1, 1), mgp = c(3, 1, 0) / 1.6)
%oo <- DA.reg(Ancona, random.raters = TRUE)
%ll <- DA.reg(Ancona, random.raters = TRUE, Transform = "log")
%
%lims <- range(c(oo$Conv[, , "slope(sd)"], ll$Conv[, , "slope(sd)"]), na.rm = TRUE)
%plot(oo$Conv[, , "slope(sd)"], ll$Conv[, , "slope(sd)"],
%     pch = 16, xlim = lims, ylim = lims,
%     xlab = "original", ylab = "log-transformed")
%mtext("Slope of SD vs. mean", side = 3, line = 0.2)
%abline(h = 0, v = 0)
%
%plot(oo$Conv[, , "sd=K"], ll$Conv[, , "sd=K"],
%     pch = 16, xlim = 0:1, ylim = 0:1,
%     xlab = "original", ylab = "log-transformed")
%mtext("P-value for constant SD", side = 3, line = 0.2)
%@
%\begin{figure}[tb]
%  \centering
%  \includegraphics[width=0.98\textwidth]{Ancona-trsf}
%  \caption{Checking the effect of transformation. Left panel: Slopes
%    of the regression of SD on averages. Right
%    panel $p$-values of $H_0: \text{slope}=0$ for the regression.}
%\label{fig:trsf}
%\end{figure}
%From Figure \ref{fig:trsf} it is very clear that the log-transform of
%the data provides a much better fit of the model.
%This is easily accommodated using the \code{Transform}
%argument to \code{BA.est}:
The limits of agreement for the log-transformed data is
1.01. Since this estimate
is based on the standard deviations of the log-transformed data the
estimated variance components essentially represent coefficients of
variation, see \eg, \citet[chapter 9]{Carstensen.2010b}. To convert this
to a sensible number applicable on the original (count) scale, we take
the exponential function of these values to get a multiplicative
factor.
% Thus in the case of the log-transformed data the limits of
% agreement will be a multiplicative factor, 
%
The limits of agreement on the log scale is
$\pm$1.01, so the ratio
between the score from two raters (largest/smallest) is with 95\%
probability less than  $\exp(1.01) = 2.75$ for
assessments of the number of points the same picture.


%\subsubsection{Visualization of the LoA}
We can illustrate the relationship between the two transformations
(identity or log) by making a graph for converting between two random
raters. The mean conversion line will necessarily be the identity
line, so the plot will basically show envelopes of where we can expect
to find observations. We add points from random pairs of raters so we can
compare the estimated envelopes to the observed data.

% XXXX hertil
Also note that the envelope based on the log-transformation is with
straight lines because we have a model where the conversion between
methods (raters) is forced to go through 0. The result is shown in
Figure~\ref{fig:envelopes}, where it is clear, that the analysis based
on the log-transformed data (limits shown with the dashed lines in
Figure~\ref{fig:envelopes}) captures the variation and differences
among raters much better than the untransformed data (the solid
lines) for all count levels.
\begin{figure}%[tb]
  \centering
  \includegraphics[width=0.8\textwidth]{Ancona-envelopes}
  \caption{Visualization of the LoA from the Ancona data. The
    solid lines are from a model using original count data, the dashed
    lines from a model for log-transformed data.
    %Dotted lines are from
%    using the median to summarize the variances of the raters instead
 %   of the mean (full lines).
    The points are from 16 randomly chosen
    pairs of raters.}
\label{fig:envelopes}
\end{figure}

%We can of course show the same conversion on a plot of the differences
%versus the averages; it is merely a $45^\circ$ rotation of the plot. The
%conversion can be done using the \code{y2DA} function, which
%converts pairs of observations to differences and averages. The result
%is shown in Figure~\ref{fig:DA-envelopes}.
%<<DA-envelopes,fig=TRUE>>=
%par(mar = c(3, 3, 1, 1), mgp = c(3, 1, 0) / 1.6, las=1)
%plot(NA, xlim = c(0, 200), ylim = c(-100, 100),
%         xlab = "Average count", ylab = "Count difference", xaxs="i", yaxs="i")
%abline(v = seq(0, 200, 10), h = seq(-100, 100, 10), col = gray(0.8))
%for(pm in c(-1, 1)) {
%  abline(pm * LoA.mn, 0, lwd = 2, lty = 1, col = "black")
%  abline(0, y2DA(0, exp(pm * LoA.log.mn))[2], lwd = 2, lty = 1, col = "blue")
%}
%for (i in sample(levels(Ancona$meth), 16)) {
%  for (j in sample(setdiff(levels(Ancona$meth), i), 1)) {
%    wd <- to.wide( subset( Ancona, meth %in% c(i,j) ) )
%    ro <- sample(0:1, 1)
%    points((wd[, 3 + ro] + wd[, 4 - ro]) / 2,
%            wd[, 3 + ro] - wd[, 4 - ro],
%            col = clrs[wd$repl], pch = 16, cex = 0.5)
%  }
%}
%@ % def
%\begin{figure}[tb]
%  \centering
%  \includegraphics[width=0.8\textwidth]{Ancona-DA-envelopes}
%  \caption{Visualization of the LoA from the Ancona data. The
%    black lines are from a model using original count data, the blue
%    lines from a model for log-transformed data. The points are
%    observations from 16 randomly chosen pairs of raters, and the
%    colouring of the points refer to the replicates.
%  }
%\label{fig:DA-envelopes}
%\end{figure}



\subsubsection{Repeatability}

% \subsubsection{Additive effects}

% latex table generated in R 3.2.3 by xtable 1.8-0 package
% Tue Jan  5 15:29:14 2016
\begin{table}[t]
\centering
\caption{Individual repeatability coefficients ($2.8\sigma_m^2$) for the 17 different raters.} 
\label{ratertable}
\begin{tabular}{rrr}
  \hline
  \hline
33.9 & 29.5 & 46.6 \\ 
  30.9 & 48.0 & 43.8 \\ 
  57.7 & 36.2 & 122.4 \\ 
  34.8 & 52.9 & 53.7 \\ 
  52.4 & 23.6 & 88.0 \\ 
  46.2 & 41.5 &  \\ 
   \hline
\end{tabular}
\end{table}
The repeatability coefficient for rater $m$ is calculated as $2
\times \sqrt{2 \sigma_m^2}\approx 2.8\sigma_m$ provided we assume
exchangeability for the replicates.
For the Ancona data the individual estimated repeatability coefficients on the
original scale are the $17$ values shown in Table~\ref{ratertable} and
the mean repeatability coefficient is
49.53.  Broadly speaking we
can say that the repeatability of ``academic point-counters'' (such as
represented by the course participants at the SISMEC conference in
Ancona) is 49.53 points,
meaning that assessing the same picture twice by the same randomly
chosen rater will
produce two guesses closer than about 50 with probability 95\%.






If replicates are linked we should include the between-replicates
variation in the calculation of the repeatability, if this variation
is considered unavoidable (and universal) between replicates.  In the case of the
Ancona experiment, it is debatable whether replicates consisting of
showing a rotated version of the picture are replicates in the
repeatability sense. But if this is considered so, then we should
calculate the individual repeatability coefficients using
\eqref{eq:linked-repeat} and compute the mean repeatability
coefficient among raters which then becomes
50.23.  The mean
repeatability is virtually the same for the Ancona dataset whether we
consider the replicates as exchangeable or as linked.

% \subsubsection{Multiplicative effects}

We saw that there was some indication that data was better described
by a constant variance model for the log-transformed counts. In that
case we should compute the repeatability on the multiplicative scale,
\ie we should basically do the same calculations on the
log-transformed data.  Again, these are (``expected'') upper limits of
absolute differences of natural-log transformed data, so not readily
interpretable. If we take the exponential of these, we will get upper
limit of a 95\% prediction interval for the \emph{ratio} of the larger
to the smaller rating between two replicates.
Hence, assessing the same picture twice by the same rater will on
average produce guesses that are within a factor 2.08 of each other with
probability 95\%.


All computations shown here are available as a detailed \R script with
additional comments at \\ \verb+www.biostatistics.dk/agreement/ancona.pdf+.


\subsection{Consumer evaluation of chocolate}


% \usepackage[utf8]{inputenc}


Sensometrics is typically concerned with quantifying differences
between a set of brands or products. The brands are typically scored
on several features simultaneously, and the scores are aggregated or
analyzed with a multivariate model. If we focus on just a single feature
then the products or brands are the items and the consumers that rate
the products are the raters. Note that the objective is different from
traditional sensometrics: We are interested in evaluating how two
randomly picked consumers rate the same product.

The data used here refer to six varieties of chocolates sold in
France. Each chocolate was evaluated on a structured scale from 0 to
10, by 222 consumers, according to their liking (0) or disliking (10)
\citep{SensoMineR}. For our purpose the six varieties of chocolate will
act as items and the 222 consumers as raters and we treat the score as
a continuous outcome. Each consumer has only rated each chocolate
brand once so there are no replicates. The data can be found in the
\pkg{SensoMineR} package for \R.

An overview of the data can be seen from a violin plot of the scores
for each item (chocolate brand) as shown in
Figure~\ref{fig:choco-violin}.  The densities in
Figure~\ref{fig:choco-violin} suggest that the consumers have very
varying opinion on the chocolate scores for all the brands, and that
there are no clear substantial difference in the brand scores. The
plot also indicates that it may be problematic to get to random
consumers to agree on the chocolate quality for any brand.


\begin{figure}%[tbp]
  \centering
\includegraphics[width=0.9\textwidth]{choco-violin}
\caption{Illustration of the chocolate data. Each line shows a violin
  plot and the actual scores for a particular chocolate brand. The
  scores are shown with a slight jitter along the $x$ axis which is
  why scores below 0 and above 10 appear in the plot.}
\label{fig:choco-violin}
\end{figure}

% We can get estimates of the variance components using the
% \code{BA.est} function on the \code{Meth} object \code{Ancona},
% where the \code{random.raters} argument is set to \code{TRUE} in
% order to consider methods/raters as random effects instead of fixed
% effects:
% XXXX skal det her med igen?
% What is printed by the last command is merely the estimated variance
% components using the model where the conversion between methods have
% intercept is 0 and slope 1. 

% is stored in the \code{LoA} element returned by
% \code{BA.est}, so the limits are (which we save for later use)

The limits of agreement between two random consumers is
6.66. In other words there is a
95\% probability that two randomly selected consumers differ less than
6.7 points in their assessment
of the score of a chocolate. Since the scores range from zero to ten
the limits of agreement must be said to be very wide.


% \subsubsection{Repeatability}


\begin{figure}%[tbp]
  \centering
\includegraphics[width=0.9\textwidth]{choco-choco-vars}
\caption{Individual repeatibility coefficients for the chocolate data.}
\label{fig:choco-vars}
\end{figure}


Figure~\ref{fig:choco-vars} shows the estimated individual residual
standard deviation for each rater and it is clear that there is a
substantial heterogeneity in the individual variances. Essentially we
could compute the repeatability using
\eqref{repeatability-coefficient} if we assume exchangeability, but
since we have no actual replicates of measurements on the same item in
this dataset we refrain from computing it.




\section{Discussion}
%Often it is of interest to assess the similarity between different
%methods of measurement, and when measurements
%are taken with different methods we wish to determine whether the methods
%agree with one another, or rather, quantify the degree of disagreement
%in order to be able to make an informed substantive judgment.

In this paper we have extended the modeling approach of
\citet{Carstensen.2008b} to address the question of precision when the
methods represent a random sample of observers (raters) that are asked
to produce a measurement. In a sense this is analogous to the binary
case that was considered in \citet{fleiss:etal:2004}.  Our extension
incorporates the methods as random effects in the model and in that
vein, the prediction of the difference between measurements by two
random raters on a future item will inevitably be a prediction that
has zero mean, and where only the variation between and within raters
is involved. The proposed modeling approach can handle individual
skills of the raters as well as it accommodates designs where not
every rater necessarily needs to assess every item.

The calculation of the limits of agreement relies on the law of total
variance so it should be noted that the approximations used may be
rather crude when the number of available raters is small. It should
be emphasized that predictions of differences between future randomly
chosen raters is highly dependent on the assumption that the sample at
hand is actually representative and that there is no way to test this
assumption.

\citet{Choud2008} presents the methodology for computing limits of
agreement based on tolerance intervals instead of the traditional
prediction intervals that we also use here. Tolerance intervals are
more desirable for limits of agreement than prediction intervals
because we can attach a confidence level to the prediction interval so
we have better control over the actual coverage level instead of just
an average asymptotic level.  \citet{Bland:Altman.1999} also suggest
to use tolerance limits for limits of agreement with small sample
sizes. The prediction and tolerance intervals will be identical when
the number of observations increases so the benefit of the tolerance
interval is largest for small samples. There are two reasons that we
do not consider tolerance intervals in this manuscript. First we need
large samples anyway to obtain useful estimates of the variance
components in model \eqref{mod-linked}.  Second, in order to obtain
tolerance intervals --- even for small samples --- a double bootstrap
approach is needed for the computations. Unless the experimental
design is balanced a lot of extra assumptions on resampling units and
exchangeability is needed for the bootstrap approach to be
valid. Tolerance intervals are preferred, but we use the prediction
intervals since we have already made the implicit assumption that we
have a sufficiently large sample for the estimates to be useful
through prediction intervals.


%% The empirical mean of rater variances used in \eqref{approximation} is
%% obviously a good estimator of the theoretical mean variance when the
%% number of available raters is large. For smaller number of raters one
%% might be inclined to use a more robust estimator of the central
%% tendency as for example the median.  However, while the mean and the
%% median will give the same results if the distribution of rater
%% variances is symmetric, the two summary measures will differ when the
%% distribution of the residual variances is asymmetric. Since we seek to
%% estimate the mean of the distribution of residual variances, we would
%% introduce a bias in our computations and the bias will persist even if
%% the number of raters increases.

%% \SweaveInput{median}

%% As an example, consider a situation with two types of raters: those
%% with high precision (small variance) and those with low precision
%% (large variance) and assume the two types of raters are represented in
%% the population at a ratio of 3:1, say. Then the median will match the
%% variance of the most prevalent group and will consistently provide a
%% wrong estimate of the mean within-rater variance.

%% This problem is illustrated in Figure~\ref{fig:dists} where we have
%% simulated data using three different distributions for the residual
%% variances within raters: a uniform distribution, a skewed distribution and a
%% bimodal distribution. For each distribution we have looked at
%% scenarios where an increasing number of raters are randomly selected
%% from a population, and for each scenario made 500 simulated selections
%% from the distribution of rater-variances.
%% %
%% We have shown the distribution of 500 simulated means and medians of
%% the rater variances for different sizes of the random sample of raters
%% (2--40). The distributions are shown using the 2.5, 50 and 97.5\%
%% quantiles of the simulated means of variances.

%% The uniform distribution is symmetric and the mean and median produce
%% similar results. For the skewed distribution we get unbiased estimates
%% for the mean but biased (albeit more precise) estimates for the
%% median. For the bimodal distribution we get a consistent bias for the
%% median, which clearly underestimates the true mean. It is an
%% \emph{under}estimate because the largest chunk of the variances are
%% the smaller ones. There are two noticeable drops in the upper limits
%% of the quantiles for the median estimates for the bimodal distribution
%% between 8 and 10 raters and between 16 and 18 raters. These marked
%% drops are artifacts due to the combination of applying the median
%% together with a bimodal distribution. There are essentially two types
%% of raters, and when the number of raters is low we may sometimes ---
%% simply due to random sampling --- obtain the majority of observations
%% from groups with the largest variance. When the number of raters
%% increases, the probability of having the majority of raters from the
%% smallest falls below 2.5\% and the bands stabilizes.

%% Hence, the mean is still preferable as an estimator to more classical
%% robust measures like the median. But 


% \clearpage

\nocite{R}
%\nocite{Epi}
%\nocite{lattice}
\nocite{MethComp}


%\bibliographystyle{wileyj}
\bibliographystyle{imsart-nameyear}
\bibliography{localrefs}

% \listoffigures



\end{document}
